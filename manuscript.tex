\documentclass[]{aiaa-tc}% insert '[draft]' option to show overfull boxes

\usepackage{aiaa_packages}

 \title{Low-Thrust Trajectory Design Near Asteroids Using Reachability Sets}

 \author{
  Shankar Kulumani\thanksibid{1}%
    \thanks{Doctoral Student, \href{mailto:skulumani@gwu.edu}{skulumani@gwu.edu}. Student AIAA Member.}
  \ and Taeyoung Lee\thanksibid{2}\thanks{Associate Professor, \href{mailto:tylee@gwu.edu}{tylee@gwu.edu}. AIAA Member}\\
  {\normalsize\itshape
   Mechanical and Aerospace Engineering, George Washington University, 800 22nd St NW, Washington DC }\\
   }

 % Data used by 'handcarry' option if invoked
 \AIAApapernumber{YEAR-NUMBER}
 \AIAAconference{Conference Name, Date, and Location}
 \AIAAcopyright{\AIAAcopyrightD{YEAR}}

\begin{document}

\maketitle

\begin{abstract}
Write an awesome abstract

Use the reachability set to compute transfers

extend work already done in three body problem to motion around asteroid

Numerical example for asteroid 4769 Castalia
\end{abstract}

\section*{Nomenclature}

\begin{tabbing}
  XXX \= \kill% this line sets tab stop and adds a newline
  $J$ \> Jacobian Matrix \\
  $f$ \> Residual value vector \\
  $x$ \> Variable value vector \\
  $F$ \> Force, \si{\newton} \\
  $m$ \> Mass, \si{\kilo\gram} \\
  $\Delta x$ \> Variable displacement vector \\
  $\alpha$ \> Acceleration, \si{\meter\per\second\squared} \\[5pt]
  \textit{Subscript}\\
  $i$ \> Variable number \\
\end{tabbing}

\section{Introduction}

% Motivation for missions/studying asteroids
Small solar system bodies, such as asteroids and comets, are of significant interest to the scientific community.
These small bodies offer great insight into the early formation of the solar system.
This insight offers additional detail into the formation of the Earth and also the probable formation of other planetary systems.
Of particular interest are those near-Earth asteroids (NEA) which inhabit heliocentric orbits in vicinity of the Earth.
These easily accessible bodies provide attractive targets to support space industrialization, mining operations, and scientific missions..
NEAs potentially contain many materials useful for propulsion, construction, semiconductors, precious and strategic metals~\cite{ross2001}.
In addition, these NEAs are also of concern for their potential to impact the Earth~\cite{wie2008}.
Asteroids and comets are the greatest threat to future civilizations and as a result there is a focused effort to mitigate these risks.
In spite of the great interest in asteroids, the operation of spacecraft in thier vicinity is a challenging problem.

% Difficulty in system model 
While there has been significant study of interplanetary transfer trajectories, relatively less analysis has been conducted on operations in the vicinity of asteroids.
The dynamic environment of around asteroids is strongly perturbed and challenging for analysis and mission operation.
Due to their small size and therefore low gravitational attraction asteroids have irregular shapes and potentially chaotic spin states.
Furthermore, due to the low gravitational attraction, other non-gravitional effects, such as solar radiation pressure, become more significant.
The orbital enviornment is in general quite complex and analytical insights are difficult to generate.

% Must estimate gravity and shape from ground based optical measurements
% most use a spherical harmonic model or a ellipsoid model but we use a polyhedron model
An accurate gravitional potential model is neccessary for the operation of spacecraft about asteroids.
Additionally, a detailed shape model of the asteroid is needed for trajectories passing close to the body.
The classic approach is to expand the gravitational potential into a harmonic series and compute the series coefficients.
Radio tracking data of an orbiting spacecraft allows one to estimate the series coefficients.
The harmonic representation are guaranteed to converge outside of the circumscribing sphere and can be truncated at a finite order based on accuracy requirements.
However, the harmonic expansion is always an approximation as a result of the infinite order series used in the representation.
Additionally, the harmonic model used outside of the circumscribing sphere is not guaranteed to converge inside the sphere.
As a result, additional steps must be taken to determine a valid harmonic expansion for particles close to the body. 
This results in a cumbersome gravity field model that requires additional constraints to ensure continuity and validity at the radius of the circumscribing sphere~\cite{werner1996}.
Instead, we model the asteroid as a constant-density polyhedron and avoid the issues of the harmonic expansion while enabling several other useful advantages.

% benefits of the polyhedron model
A polyhedral model of the surface of an asteroid can be determined from remote optical or radar sensors.
The faces of the polyhedron can be large or small and allows for fine detail such as depression, craters, ridges, or interior voids. 
In addition, there is no requirement for the body to be modeled at a uniformly high resolution so small details can be incorprated with minimal cost.
From the shape model, an analytical, closed form expression for the gravitational potential can be derived.
The polyhedral approach provides an accurate gravitational model consistent with the resolution of the shape and the chosen discretization.
Furthermore, the polyhedron model is an exact solution up to the surface of the body. 
Therefore, this model is ideal for missions traversing large region both close and far from the asteroid.

% similarity to the three body problem
The irregular shape of asteroids leads to a richer dynamics as compard to the familar Kepler's problem, which is based on a point mass assumption.
The motion of particles around asteroids is modeled using the restricted full two-body problem. 
This model has many similarities with the restricted three body problem, and much of the theory developed there is also applicable.
There has been much work in characterizing the relative equilibria and periodic orbits in the three-body problem.
In addition, orbital transfers have been designed which exploit the invariant manifolds of the three-body problem for low-energy manuevers~\cite{mingotti2011,grebow2011}.
The similarity of the three-body problem to motion around astreroids allows for much of the same theory to be applied to the asteroid problem.
More recently, the invariant manifolds around asteroids have been proposed for orbital transfers and landing manuevers~\cite{mondelo2010,herrera2014}.

% use of low thrust propulsion

The application of optimal control methods for orbital trajectory design is nontrivial.
In order to implement any optimization method a sufficiently accurate ``initial guess'' is required.
Frequently, insight into the problem or intuition on the part of the designer is often required to determine initial conditions that will converge to the optimal solution.
However, the asteroid system dynamics are nonlinear and exhibit chaotic behaviors. 
This makes solving the optimization problem highly dependent on the initial condition.
Similar to the three-body problem, there is an insufficient number of analytical constants to derive an analytical solution in general. 
As a result, accurate numerical methods are required to determine optimal solutions.
These methods are critically dependent on accurate initial guess in order to allow for convergence to an optimal solution.

% SYSTEMATIC DESIGN TO AVOID DIFFICULTIES IN CHOOSING A GOOD INITIAL GUESS
% CAPTURE LONG TERM EFFECTS OF LOW THRUST ACCURATELY IN NUMERICAL SIMULATION
In this paper, we extend the design method developed in the three-body problem to motion about asteroids~\cite{kulumani2015}.
Our approach avoids the difficulties in selecting an appropriate initial guess for optimization.
Instead, we utilize the concept of the reachability set to enable a simple methodology of selecting initial conditions to achieve general orbital transfers. 
Given an initial condition and fixed time horizon, the reachable set is the set of states attainable, subject to the operational constraints of the spacecraft. 
In addition, the generation of the reachable set allows for a more systematic method of determining initial conditions and eases the burden on the designer.
This simple methodology allows for extended transfer trajectories which iteratively approach a desired target orbit.


\section{Asteroid Model}

\subsection{Shape Model}
Cite location of shape model database

Cite Scheeres orbits close to 4769 Castalia

Assuming constant rotation rate for asteroid

\subsection{Polyhedron Gravity Model}

The gravitational attraction due to a constant density polyhedron is given as
\begin{align*}
	U(\vecbf{r}) &= 1\frac{1}{2} G \sigma \sum_{e \in \text{edges}} \vecbf{r}_e \cdot \vecbf{E}_e \cdot \vecbf{r}_e \cdot L_e - \frac{1}{2}G \sigma \sum_{f \in \text{faces}} \vecbf{r}_f \cdot \vecbf{F}_f \cdot \vecbf{r}_f \cdot \omega_f 
\end{align*}
where \( \vecbf{r}_e, \vecbf{r}_f \) is the vector from the spacecraft to any point on the respective edge or face.
The polyhedron is defined by an array of vectors in the body fixed frame which define the vertices of the body.
Triangular faces are defined as the plane created by any three vertices. 
Consider three vectors \( \vecbf{v}_0, \vecbf{v}_1, \vecbf{v}_2 \), assumed to be ordered in a counterclockwise direction, which define a face. 
The outward normal vector to the face and to each edge is defined as
\begin{align*}
	\hat{\vecbf{n}}_f &= \parenth{\vecbf{v}_{1} - \vecbf{v}_0} \times \parenth{\vecbf{v}_{2} - \vecbf{v}_1} ,\\
	\hat{\vecbf{n}}_{e} &= \parenth{\vecbf{v}_{i+1} - \vecbf{v}_i} \times \hat{\vecbf{n}}_f .
\end{align*}
Using these normal vectors we define two dyads
\begin{align*}
	\vecbf{E}_e &= \hat{\vecbf{n}}_A \hat{\vecbf{n}}_{12}^A + \hat{\vecbf{n}}_B \hat{\vecbf{n}}_{21}^B ,\\
	\vecbf{F}_f &= \hat{\vecbf{n}}_f \hat{\vecbf{n}}_f .
\end{align*}
The edge dyad \( \vecbf{E}_e \) is defined for each edge and a function of the two adjacent faces meeting at the edge. 
The face dyad \( \vecbf{F}_f \) is defined for each face and is a function of the face normal vector.

Let \( \vecbf{r}_i \) be the vector from the spacecraft to the vertex \( \vecbf{v}_i \) and it's length is \( r_i = \norm{\vecbf{r}_i} \).
The per-edge factor \( L_e \), for the edge connecting vertices \( \vecbf{v}_i \) and \( \vecbf{v}_j \), with a constant length \( e_{ij} \) is 
\begin{align*}
	L_e &= \ln \frac{r_i + r_j + e_{ij}}{r_i + r_j - e_{ij}}.
\end{align*}

For the face defined by the vertices \( \vecbf{v}_i, \vecbf{v}_j, \vecbf{v}_k \) the per-face factor \( \omega_f \) is
\begin{align*}
	\omega_f &= 2 \arctan \frac{\vecbf{r}_i \cdot \vecbf{r}_j \times \vecbf{r}_k}{r_i r_j r_k + r_i \parenth{\vecbf{r}_j \cdot \vecbf{r}_k} + r_j \parenth{\vecbf{r}_k \cdot \vecbf{r}_i} + r_k \parenth{\vecbf{r}_i \cdot \vecbf{r}_j}} .
\end{align*}

Using these definitions, the attraction, gravity gradient matrix, and Laplacian are defined as
\begin{align*}
	\nabla U ( \vecbf{r} ) &= -G \sigma \sum_{e \in \text{edges}} \vecbf{E}_e \cdot \vecbf{r}_e \cdot L_e + G \sigma \sum_{f \in \text{faces}} \vecbf{F}_f \cdot \vecbf{r}_f \cdot \omega_f ,\\
	\nabla \nabla U ( \vecbf{r} ) &= G \sigma \sum_{e \in \text{edges}} \vecbf{E}_e  \cdot L_e - G \sigma \sum_{f \in \text{faces}} \vecbf{F}_f \cdot \omega_f , \\
	\nabla^2 U &= -G \sigma \sum_{f \in \text{faces}}  \omega_f .
\end{align*}

One interesting thing to note is that both \( \vecbf{E}, \vecbf{F} \) can be precomputed without knowledge of the position of the satellite. 
They are both solely functions of the vertices and edges of the polyhedral shape model.
One a position vector \( \vecbf{r} \) is defined then the scalars \( \omega_f, L_e \) can be computed for each face and edge. 


Duplicate the equations for polyhedron model

Give an outline of how to compute the potential (which equations and order)

Cite the original werner papers

\subsection{Spacecraft Equations of Motion}

The motion of a massless particle, or spacecraft, about an asteroid shares many similarities with that of the three-body problem.
As is typical in the three-body problem, the equations of motion are usually represented in a uniformly rotating frame aligned with the two primaries.
Similarily, the equations of motion about an asteroid are also defined in a body-fixed frame with uniform rotation.
In this reference frame, the gravitational potential field is time invariant and only a function of the position of the particle.
In addition, since the rotational rate of the asteroid is constant the equations of motion are time invariant.
Finally, the use of the rotating reference frame allows for much greater insight into the dynamic structure of the behavior around the asteroid.

We define a reference frame originating at the center of mass of the asteroid. 
  

Cite scheeres and his book

Equations of motion

Jacobi constant

Zero velocity curves

Equilibrium points. Cite work done with triaxial ellipsoids and other Castalia papers

\subsection{Determining Periodic Orbits}

Periodic orbit computation summary. Cite Scheeres 2000 (Evaluation of the Dynamic environment of an asteroid)

\section{Reachability Set on \Poincare Section}
Define reachability set

Define \Poincare section

\subsection{Optimal Control Formulation}

Formulate OCP to generate the reachability set

Discuss the constraints and angles on the four dimensional space of the section

\section{Numerical Simulation}

Transfer about asteroid Castalia - cite some Scheeres papers

Plots of \Poincare section

Full trajectory plots

Thruster magnitude is chosen to emulate some real thrusters - cite the Dawn paper

\section{Conclusion}


\section*{Acknowledgments}

A place to recognize others.

\bibliographystyle{aiaa} 
\bibliography{library}

\end{document}
